\documentclass[11pt]{report}
\usepackage[a4paper]{geometry}
\usepackage[myheadings]{fullpage}
\usepackage{fancyhdr}
\usepackage{lastpage}
\usepackage{graphicx, wrapfig, subcaption, setspace, booktabs}
\usepackage[T1]{fontenc}
\usepackage[font=small, labelfont=bf]{caption}
\usepackage[protrusion=true, expansion=true]{microtype}
\usepackage{sectsty}
\usepackage{url, lipsum}
\usepackage{mathptmx}
\usepackage{listings}
\usepackage[utf8]{inputenc}
\usepackage[francais]{babel}
\pagestyle{plain}

\newcommand{\HRule}[1]{\rule{\linewidth}{#1}}
\onehalfspacing
\setcounter{tocdepth}{5}
\setcounter{secnumdepth}{5}

\renewcommand{\thesection}{\arabic{section}}


%-------------------------------------------------------------------------------
% TITLE PAGE
%-------------------------------------------------------------------------------

\begin{document}

\title
{
	\Large{Projet de Réseau}
	\HRule{2pt} \\ [0.5cm]
	\LARGE \textbf{\uppercase{Jeu du Morpion Aveugle}}
	\HRule{2pt} \\ [0.5cm]
	\normalsize \today
}

\date{}

\author
{
	\LARGE{Université de Bordeaux} \\
	\\
    Kenji FONTAINE \\
    Enzo PERUZZETTO \\
}

\maketitle

%-------------------------------------------------------------------------------
% Présentation du projet - Introduction
%-------------------------------------------------------------------------------

\section{Présentation du projet}

Ce travail a été réalisé dans un cadre universitaire par des étudiants en
Licence informatique. \\
Le jeu de morpion est un jeu à 2 joueurs se déroulant sur une grille de
3 cases sur 3. Chaque joueur à tour de rôle marque une case dans la grille.
Le premier joueur parvenant à aligner 3 de ses marques est déclaré gagnant.
Si la grille est complétement remplie sans qu'il n'y ait 3 marques identiques
alignées, il y a match nul. \\
Le jeu du morpion aveugle est une variante du jeu de morpion dans laquelle
les joueurs ne voient pas les coups joués par leur adversaire respectif.
Si un joueur essaie de jouer une case qui a déjà été jouée, il est
informé que cette case est prise par son adversaire et doit en jouer une autre. \\
Le but de ce projet est d'implémenter ce jeu du morpion aveugle en réseau.

\section{Lancement d'une partie}

Nous avons besoin de 3 terminaux pour lancer une partie, 1 serveur et 2 clients.
Se placer dans le dossier src et lancer les commandes suivantes : \\
- python3 serveur.py \\
- python3 client.py 8888 \\
- python3 client.py 8888 \\
Le fichier client.py prend en paramètre le port du serveur. Il est codé en dur
dans le fichier serveur.py à la ligne 14 et peut être modifié. \\
Une fois la partie terminée, le serveur fermera et les clients s'arrêteront.


%-------------------------------------------------------------------------------
% Les fichiers
%-------------------------------------------------------------------------------

\newpage

\section{Fonctonnalités des fichiers}

\subsection{client.py}

Il s'agit du fichier exécuté par le client se connectant au serveur pour
jouer une partie. Il prend en paramètre le port du serveur en question.
Le client crée un socket pour se connecter au serveur puis lance une boucle
infinie dans laquelle il va recevoir les informations envoyées par le serveur.
Les informations reçues sont des octets qui ont été générées en utilisant pickle
côté serveur. \\
Le client réutilise pickle pour retransformer les octets afin d'être utilisables.
La fonction handle est alors appelée, elle a un comportement différent
en fonction du code porté par les données retransformées. \\
La structure des données envoyées par le serveur est expliquée plus en détails
dans la section dédiée au fichier \textit{macro.py}.

\subsection{serveur.py}

Il s'agit du fichier exécuté par le client hébergeant la partie du morpion
aveugle. Le serveur commence par crée un socket et dans une boucle infinie,
accepte les connexions des utilisateurs.
Une fois 2 utilisateurs connectés, le serveur lance la partie en appelant la
fonction main du fichier \textit{jeu.py}.

\subsection{jeu.py}

Ce fichier permet au serveur de lancer une partie du morpion aveugle.
Grandement inspiré du fichier \textit{main.py} qui était fournis, il a été modifié
afin de gérer 2 joueurs et les différents envois d'information faits par le
serveur. \\
Le détail relatif aux informations envoyées par le serveur est donné dans la
section suivante.

\subsection{macro.py}

Ce fichier contient les différentes informations que le serveur envoie à ses
clients. On y trouve la définition de plusieurs macros ainsi que de leur
"message" respectif. \\
Les données envoyées sont stockées dans un tableau de la
forme suivante : [code, données]. \\
Où le code est un entier qui va déterminer le comportement de la fonction
handle du client (ligne 12, client.py)
et les données seront à traiter par le client. Le serveur va ensuite utiliser
pickle pour transformer les tableaux en octets afin de pouvoir les envoyer
au travers des sockets. \\
Les macros de la forme MSG\_XXXXX sont les octets générés par pickle,
ceux-ci permettent d'alléger le code du serveur.

\subsection{grid.py}

Ce fichier permet d'afficher, de modifier la grille utilisée par un jeu du
morpion. Il y a eu quelques petites modifications, telles que le renommage de
la classe grid (grid -> Grid) ou la modification du tableau des symboles pour
mieux coller avec notre code.

%-------------------------------------------------------------------------------
% Difficultés rencontrées
%-------------------------------------------------------------------------------

\newpage

\section{Difficultés rencontrées à l'implémentation}

Au début, nous pensions tout faire faire par le serveur. Héberger la partie,
gérer les coups, modifier les grilles. Le serveur ferait alors le rendu des
grilles (fonction display()) et envoyaient les grilles aux joueurs sous forme
de chaîne de caractères. Le client devait juste taper le nuéro de la case
qu'il souhaitait jouer, quand il lui était demandé.
Cependant, il était dit dans le sujet que le client devait faire le rendu
de la grille, donc nous avons abandonner cette idée.

\vspace{5mm}
\par

Dans une seconde version, notre serveur communiquait avec le client en envoyant
uniquement des entiers, sans utiliser pickle. Ces entiers modifiaient le
comportement de la fonction handle du client. Le code était comparable à
celui de la version actuelle. La différence majeure étant qu'au lieu d'envoyer
des tableaux en utilisant pickle, le serveur envoyait des entiers convertis
en octets en faisant des cast. \\
Le fichier \textit{macro.py} contenait par exemple des lignes de cette forme ;

\begin{verbatim}
[...]
WELCOME = 10
[...]
MSG_WELCOME = bytes(str(WELCOME).encode('utf')))
[...]
\end{verbatim}

Tout marchait bien, sauf l'envoie de grille du serveur au client.
L'implémentation n'était pas stable et l'envoie de grille échouait
2 fois sur 3. Nous n'avons pas réussi à identifier la source du problème et
avons finalement abandonner cette version.

\vspace{5mm}
\par

Dans notre troisième version, qui est la version actuelle, l'objectif était
de reprendre la seconde version et de changer la façon dont le serveur
communiquait avec les clients. Il fallait qu'on puisse envoyer un entier ainsi
que des données pour pouvoir faire l'envoi de grille. C'est là que nous avons
utiliser pickle qui permettait de simplement convertir des tableaux en octets
ou des octets en tableaux.

%-------------------------------------------------------------------------------
% Extensions
%-------------------------------------------------------------------------------

\section{Extensions}

\subsection*{Partie seul}

Si vous lancez le fichier client.py sans paramètre vous pourrez jouer seul
contre une IA jouant des coups aléatoires.

\subsection*{Mode spéctateur}

c'est trop dur :(

%-------------------------------------------------------------------------------
% Difficultées rencontrées Extensions
%-------------------------------------------------------------------------------

\newpage
\section{Difficultées rencontrées aux extensions}

%-------------------------------------------------------------------------------
% Conclusion
%-------------------------------------------------------------------------------

\newpage

\section{Conclusion}

\end{document}
